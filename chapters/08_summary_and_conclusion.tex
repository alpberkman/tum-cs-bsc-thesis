% !TeX root = ../main.tex

% Summary and Conclusion
% First summarize the overall work
% Explain the key findings and impact
% Mention the source code
\chapter{Summary and Conclusion}\label{chapter:summary_and_conclusion}
In this thesis, we have strived to build an add-on for a verifier that checks the correctness of emulators.
This add-on, which is also called the reproducer, was designed to replicate bugs that the Focaccia verifier detects.
The work on the reproducer mostly went on transforming the output of the verifier and turning it into assembly instructions.
This entailed preparing the memory, stack, and registers.
We had to deal with symbolic expressions and use them to produce code that could trigger the detected bugs.
We have built a generic interface that should be able to extract the aforementioned values and an x86-specific program that could print x86 assembly code.

At the start of our project, we had the incorrect assumption that most of the translation bugs stemmed from incorrect execution path.
However, we came to notice that in fact most of the errors stem not from changes in the execution path but rather from incorrect implementation of general instructions.
This revelation led us to adjust the reproducer accordingly.

We have tested our reproducer on real bugs and managed to create a binary that can trigger the same erroneous behavior while being tiny compared to the original program.
This binary is only one-sixth the size of the original and its symbolic trace is one 57th of the original trace.
Even the emulator log is only one 122th of the original log.

We started this project in the hopes that we might create a useful tool that can help emulator developers pinpoint bugs and make reproducing them easier.
Although our project hasn't seen any real-world usage we have managed to single out existing buggy instructions and reproduce them.
We hope that the reproducer can be of help in discovering bugs.
The source code of this project can be found in \url{https://github.com/TUM-DSE/focaccia} under the reproducer branch.
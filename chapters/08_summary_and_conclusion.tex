% !TeX root = ../main.tex

% Summary and Conclusion
% First summarize the overall work
% Explain the key findings and impact
% Mention the source code
\chapter{Summary and Conclusion}\label{chapter:summary_and_conclusion}
In this thesis, we have strived to build a reproducer add-on for the Focaccia verifier.
This verifier checks the correctness of emulators, while our add-on replicates bugs that were detected.
The work on the reproducer mostly involved transforming the verifier's output into an assembly program.
This process entailed using symbolic expressions to pinpoint necessary values and then setting up the memory, stack, and registers accordingly.
We have built a generic interface that should be able to extract the aforementioned values and an x86-specific program that could print the respective assembly code.

At the start of our project, we had the incorrect assumption that most of the translation bugs stemmed from incorrect execution path.
However, we noticed that most of the errors stem not from changes in the execution path but rather from incorrect implementation of general instructions.
This revelation led us to adjust the reproducer accordingly.

We have tested our reproducer on actual bugs and created a binary that can trigger the same erroneous behavior while being tiny compared to the original program.
This binary is only one-sixth the size of the original, and its symbolic trace is one 57th of the original trace.
Even the emulator log is only one 122th of the original log.

We started this project to create a useful tool that can help emulator developers pinpoint bugs and make reproducing them easier.
Although our project has yet to see any real-world usage, we have managed to single out existing buggy instructions and reproduce them.
We hope that the reproducer can help discover bugs.
The source code of this project can be found in \url{https://github.com/TUM-DSE/focaccia} under the reproducer branch.
The final version can be downloaded from \url{https://github.com/TUM-DSE/focaccia/releases/tag/alp-berkman-thesis-final}.

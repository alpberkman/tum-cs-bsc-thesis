% !TeX root = ../main.tex

% Implementation
% Explain all low-level implementation level details
\chapter{Implementation}\label{chapter:implementation}
In this chapter we will go over low-level implementation details and explain what we had to do in order get the reproducer working.
These include some minor additions to the verifier, a basic algorithm to find and set the memory values and some assembly stubs both to prepare the state and run the executable.
All of these additions are made in python as the verifier is written in it.

\section{Additions to the Focaccia}
Focaccia is a full fledged verifier that can find the bugs in emulators.
However because it is desinged to just be the verifier, it lacks some usefull features that the reproducer needs.
For example it can return 

actual merging some assembly details?
what I get from the verifier, symbolic expression?
address matching, assembly, syscalls
my additions to Focaccia

\section{Data Section}
\section{Code Section}
\subsection{Stubs}
\subsection{Instructions/Basic Block}
\subsection{Register Setup}
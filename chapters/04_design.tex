% !TeX root = ../main.tex

% Design
% Explain the system design in detail
% Explain each system component sufficient details
% Explain workflows using algorithms
\chapter{Design}\label{chapter:design}



The following chapter presents the core design decisions that were used to extend the Focaccia verifier with the reproducer.
Extra attention is given to the reproducer interface and the data that is taken from the verifier and how it relates to the whole verifier.
When trying to understand some design choices, it is important to keep in mind that the reproducer was deisgned as an add-on.

\section{Focaccia Interface}
The reproducer comes in the last section of the verifier.
This means it is run after verification is done and errors are found.
When running the reproducer the higher level interface gets two inputs.

These are:
\begin{itemize}
    \item Minimum error severity
    \item Result of verification including:
    \begin{itemize}
        \item pc
        \item txl
        \item ref
        \item errors
        \item snap
    \end{itemize}
\end{itemize}

\section{Design of the Reproducer}

\subsection{Instructions and Basic Blocks}

\subsection{Registers}

\subsection{Memory}

The reproducer is a direct extension for the verifier.
Therefore it is important to first understand how the verifier works, as all of the inputs of the reproducer stem from the verifier.

map the addresses code stubs

more like overall structure

\section{Shorcommings}
In this section we will go over some instruction types that cannot be properly reproduced.
These instructions are generally flow control type instructions.